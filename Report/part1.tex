\chapter{Choix du Dataset et Sémantification}

Nous avons choisi d'utiliser un dataset concernant \href{https://data.enseignementsup-recherche.gouv.fr/explore/dataset/fr-esr-rd-moyens-administrations-type-organisme/}{les moyens consacrés à la R\&D par les administrations}.

Ce dataset est le fruit d'une enquête concernant les moyens consacrés à la R\&D réalisé par le Ministère de l'Éducation nationale, de l'Enseignement supérieur et de la Recherche en Octobre 2015. Les données onte été modifiées pour la dernière fois en octobre 2016.

Ces données sont structurées dans une table avec les headers suivants:
\begin{itemize}
    \item Le code INSEE de la région de l'organisme (codes des anciennes régions pré réforme)
    \item Le nom de la région
    \item L'année du financement
    \item Le code de l'indicateur
    \item Le label complet de l'indicateur
    \item Le code du type d'administration
    \item Le label complet du type d'administration
    \item Le code du sexe du bénéficiaire
    \item Le label complet du sexe du bénéficiaire
    \item Le code du type de personnel
    \item Le label complet du type de personnel
    \item L'état des données
    \item La valeur de l'indicateur
\end{itemize}

\vskip1cm

Pour sémantifier le dataset nous avons du l'épurer légèrement en ignorant les colonnes redondantes tels que le nom de la région (trouvable en liant notre dataset à celui de l'INSEE).

Le fichier RDF a été construit au moyen de \href{https://tarql.github.io/}{tarql}. Pour la requête Construct, se référer à \autoref{1}.

\begin{figure}[b]
\begin{lstlisting}
<2.53>  <http://www.w3.org/1999/02/22-rdf-syntax-ns#type>  dbo:Statistic ;
        igeo:codeRegion  "52" ;
        gn:name          "Pays de la Loire" ;
        dbo:creator      "4" ;
        dbo:alias        "Autres administrations" ;
        dbo:year         "2009" ;
        dc:type          "Depense interieure de R&D" ;
        dbo:status       "definitif" .
\end{lstlisting}
\caption{Exemple de tuple produit par la requête.}
\end{figure}


\chapter{Requêtes}
\section{Dataset seul}

La premiere requète a pour but d'afficher le nombre de projet subventionnés groupés par année, qui ont comme chef d'équipe une femme. On peut ainsi constater l'evolution du nombre de projet dirigés par des femmes au court des années. Voir le code ici : \autoref{req1}.


Cette seconde requete permet d'etudier le dynamisme des régions au fil des années. Le type de projet sert à differencier les secteurs, pour une étude qui pourrait être plus spécifique. Voir le code de la requete ici : \autoref{req2}.

\section{Liaison à un autre dataset}

La liaison avec d'autres datasets se fait sur les codes régions. Nous avons choisi le dataset du groupe Alapetite - Bellot - Boudine, en essayant de voir s'il y a de potentielles corélations entre l'attribution de financements par les administrations et le taux de réussite à ce concours.

\begin{verbatim}
    \autoref{req3}
    Pour exploiter la liaison de notre graph avec un autre possedant aussi des region, nous pouvons regarder les régions ayant une subvention et ayant un participant au concours étoile.
    % \label{req3}
\end{verbatim}

% A RETIRER AVANT DE RENDRE : NE PAS OUBLIER DE GROUP BY dc:type SINON LES DONNÉES VEULENT RIEN DIRE OKK

\chapter{Inférences}

Via apache fuseki nous avons pu inferé des triples et voir la modifications sur notre data set. Le changement est l'apparition de nouveau triple concernant le modèle du data set, par exemple :

\begin{lstlisting}
1	rdfs:domain rdfs:domain rdf:Property
2	rdfs:domain rdfs:range  rdfs:Class
3	rdfs:domain rdf:type    rdf:Property
4	<http://www.w3.org/2002/07/owl\#DatatypeProperty>  rdfs:subClassOf rdf:Property
5	<http://www.w3.org/2002/07/owl\#Property>  rdf:type  rdfs:Class
\end{lstlisting}

Ces tuples inferés sont le resultat de l'application des règles OWL sur les ontologies de chaque espèces de nommage (car nous avons tentés au mieux d'utilisé des nommages connus et cohérents à nos données).
Les inférences ont donc déduit des propriétés par rapport aux ontologies des propriétés du modèle de notre data set.

\chapter{Liaison au cloud Linked Data}
\section{Sélection d'un URI}

La norme Linked Data stipule que tous les URIs doivent être des URIs HTTP. Cela soulève un vrai problème dans le cas de notre dataset, car les données sont anonymisées et que le seul élément permettant de séparer deux tuples est la valeur de l'indicateur.

\chapter{Description VOiD}

La description VOiD de notre dataset et de la jointure réalisées avec d'autres datasets, qu'ils appartiennent à un autre groupe ou qu'ils soient fournis par des organismes, permet de spécifier la structure de cette jointure.

\begin{figure}[H]
\begin{lstlisting}
    @prefix rdf: <http://www.w3.org/1999/02/22-rdf-syntax-ns\#> .
    @prefix rdfs: <http://www.w3.org/2000/01/rdf-schema\#> .
    @prefix foaf: <http://xmlns.com/foaf/0.1/> .
    @prefix dcterms: <http://purl.org/dc/terms/> .
    @prefix void: <http://rdfs.org/ns/void\#> .
    
    :RFStats rdf:type void:Dataset ;
        foaf:homepage <http://research-funding.org/> .
    
    :RFStatsAdministration rdf:type void:Dataset ;
        void:subset :RFStats ;
        foaf:homepage <http://research-funding.org/administration/> ;
        dcterms:subject <http://http://dbpedia.org/ressource/Research> .
    
    :DBpedia rdf:type void:Dataset ;
        foaf:homepage <http://dbpedia.org/> .
    
    :INSEEgeo rdf:type void:Dataset ;
        foaf:homepage <http://id.insee.fr/geo/region/> .
    
    :EtoileDataset rdf:type void:Dataset ;
        foaf:homepage <http://concours-etoiles-europe.org/> .
    
    :RAnalysis rdf:type void:Linkset ;
        void:target :RFStatsAdministration ;
        void:target :DBpedia ;
        void:target :INSEEgeo ;
        void:target :EtoileDataset .
\end{lstlisting}
\caption{Description VOiD du projet.}
\end{figure}
